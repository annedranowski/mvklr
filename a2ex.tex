% !TeX root = ./examples.tex

\section{Type $A_2$}
% 
Let's look at a bunch of characters of weight $\nu = a\alpha_1 + b\alpha_2$ with $a \ge b$ because I think the results are going to be symmetric. Note, there are $\min(a,b)+1 = a+1$ Kostant partitions, MV cycles, etc. to consider.

% This is not quite true. I said a half hour ago. Because the mdegs will not be symmetric in (a,b). But maybe up to renormalization? Idk. 

\subsection{Case $(a,b) = (n,1)$} 
We always have 2 MV cycles. Their tableaux, words, mdegs are:
{\small
\[
\begin{aligned}
    &\young(1\dots12\dots22,3) & &\young(1\dots12\dots23,2) \\
    &2.11\dots1 & & 12.1\dots1 \\
    &\alpha_2 & & \alpha_1 
\end{aligned}    
\]}
Their eqms are found by normalizing their mdegs wrt to the MVy embedding:
$$\begin{aligned}
    \frac{\alpha_2}{\mdeg(0)} &= \frac 1 {(\alpha_1) (\alpha_1 + \hbar)\cdots (\alpha_1 + (n-1)\hbar) (\alpha_1 + \alpha_2)} \\
    \frac{\alpha_1}{\mdeg(0)} &= \frac 1 {(\alpha_1 + \hbar)\cdots(\alpha_1 + (n-1)\hbar)(\alpha_1 + \alpha_2)(\alpha_2)}
\end{aligned}$$
% 

\subsection{Case $(a,b) = (n,2)$} 
We always have 3 MV cycles. Their tableaux, words, mdegs are:
{\scriptsize
\[
\begin{aligned}
    &\young(1\dots12\dots22,33) & &\young(1\dots12\dots23,23) & & \young(1\dots12\dots33,22) \\
    &22.11\dots1 & & 2.12.1\dots1 & & 12.12.1\dots1\\
    &(\alpha_2)(\alpha_2 + \hbar) & & (\alpha_1)(\alpha_2) & & (\alpha_1)(\alpha_1 + \hbar) 
    % \\
    % &\frac 1 {(\alpha_1) (\alpha_1 + \hbar)\cdots (\alpha_1 + (n-1)\hbar) (\alpha_1 + \alpha_2)} & &\frac 1 {(\alpha_1 + \hbar)\cdots(\alpha_1 + (n-1)\hbar)(\alpha_1 + \alpha_2)(\alpha_2)}
\end{aligned}    
\]}
This time, to find their eqms, divide the mdegs by:
\[
(\alpha_1)(\alpha_1 + \hbar)\cdots(\alpha_1 + (n-1)\hbar) (\alpha_1 + \alpha_2)(\alpha_1 + \alpha_2 + \hbar) (\alpha_2) (\alpha_2 + \hbar)    
\]

\subsection{Case $(0,n,0)$}
% les données de Lusztig (0,n,0) et (n,0,n) ? Should correspond to the words 2121...21 and 1212...12
La donnée de Lusztig $n_\bullet = (0,n,0)$ correspond au tableau 
\[
    \young(1\dots13\dots3,2\dots2)
\]
de forme $(2n,n)$.
Le meq va etre 
{\small\[
\frac1{(\alpha_1 + \alpha_2)(\alpha_1 + \alpha_2 + \hbar) \cdots (\alpha_1 + \alpha_2 + (n-1)\hbar)(\alpha_2) (\alpha_2 + \hbar) \cdots (\alpha_2 + (n-1) \hbar)}    
\]}
since the ideal of the corresponding generalized orbital variety is (in case $n = 4$)
\[
(A_0,A_1,A_2,A_3) 
\]
in the space of block matrices of the form 
\[
    \left[\begin{BMAT}(e){cccc;cccc;cccc}{cccc;cccc;cccc} 
        0 & 1 & 0 & 0 & 0 & 0 & 0 & 0 & 0 & 0 & 0 & 0\\
        0 & 0 & 1 & 0 & 0 & 0 & 0 & 0 & 0 & 0 & 0 & 0\\
        0 & 0 & 0 & 1 & 0 & 0 & 0 & 0 & 0 & 0 & 0 & 0\\
        0 & 0 & 0 & 0 & A_0 & A_1 & A_2 & A_3 & A_4 & A_5 & A_6 & A_7\\
        0 & 0 & 0 & 0 & 0 & 1 & 0 & 0 & 0 & 0 & 0 & 0\\
        0 & 0 & 0 & 0 & 0 & 0 & 1 & 0 & 0 & 0 & 0 & 0\\
        0 & 0 & 0 & 0 & 0 & 0 & 0 & 1 & 0 & 0 & 0 & 0\\
        0 & 0 & 0 & 0 & 0 & 0 & 0 & 0 & A_8 & A_9 & A_{10} & A_{11}\\
        0 & 0 & 0 & 0 & 0 & 0 & 0 & 0 & 0 & 1 & 0 & 0\\
        0 & 0 & 0 & 0 & 0 & 0 & 0 & 0 & 0 & 0 & 1 & 0\\
        0 & 0 & 0 & 0 & 0 & 0 & 0 & 0 & 0 & 0 & 0 & 1\\
        0 & 0 & 0 & 0 & 0 & 0 & 0 & 0 & 0 & 0 & 0 & 0
        \end{BMAT}\right]
\]
defining elements 
\[
    \left(\begin{array}{rrr}
        t^{4} & 0 & 0 \\
        -A_{3} t^{3} - A_{2} t^{2} - A_{1} t - A_{0} & t^{4} & 0 \\
        -A_{7} t^{3} - A_{6} t^{2} - A_{5} t - A_{4} & -A_{11} t^{3} - A_{10} t^{2} - A_{9} t - A_{8} & t^{4}
        \end{array}\right)
\]
of MV cycles. 
More generally, the ideal will be generated by coordinates on the $n\times n$ block in position $(1,2)$ of the MVy slice.

\subsection{Case $(n,0,n)$}
La donnée de Lusztig $n_\bullet = (n,0,n)$ correspond au tableau
\[
    \young(1\dots12\dots2,3\dots3)
\]
de forme $(2n,n)$.
Le mec va etre 
{\small
\[
\frac 1 {(\alpha_1)(\alpha_1 + \hbar)\cdots (\alpha_1 + (n-1) \hbar)(\alpha_1 + \alpha_2)(\alpha_1 + \alpha_2 + \hbar) \cdots (\alpha_1 + \alpha_2 + (n-1)\hbar)}    
\]
}
since the ideal of the corresponding gneralized orbital variety is (in case $n = 4$)
\[
(A_8,A_9,A_{10},A_{11})
\]
in the space of block matrices of the form 
\[
    \left[\begin{BMAT}(e){cccc;cccc;cccc}{cccc;cccc;cccc} 
        0 & 1 & 0 & 0 & 0 & 0 & 0 & 0 & 0 & 0 & 0 & 0\\
        0 & 0 & 1 & 0 & 0 & 0 & 0 & 0 & 0 & 0 & 0 & 0\\
        0 & 0 & 0 & 1 & 0 & 0 & 0 & 0 & 0 & 0 & 0 & 0\\
        0 & 0 & 0 & 0 & A_0 & A_1 & A_2 & A_3 & A_4 & A_5 & A_6 & A_7\\
        0 & 0 & 0 & 0 & 0 & 1 & 0 & 0 & 0 & 0 & 0 & 0\\
        0 & 0 & 0 & 0 & 0 & 0 & 1 & 0 & 0 & 0 & 0 & 0\\
        0 & 0 & 0 & 0 & 0 & 0 & 0 & 1 & 0 & 0 & 0 & 0\\
        0 & 0 & 0 & 0 & 0 & 0 & 0 & 0 & A_8 & A_9 & A_{10} & A_{11}\\
        0 & 0 & 0 & 0 & 0 & 0 & 0 & 0 & 0 & 1 & 0 & 0\\
        0 & 0 & 0 & 0 & 0 & 0 & 0 & 0 & 0 & 0 & 1 & 0\\
        0 & 0 & 0 & 0 & 0 & 0 & 0 & 0 & 0 & 0 & 0 & 1\\
        0 & 0 & 0 & 0 & 0 & 0 & 0 & 0 & 0 & 0 & 0 & 0
        \end{BMAT}\right]
\]
defining elements 
\[
    \left(\begin{array}{rrr}
        t^{4} & 0 & 0 \\
        -A_{3} t^{3} - A_{2} t^{2} - A_{1} t - A_{0} & t^{4} & 0 \\
        -A_{7} t^{3} - A_{6} t^{2} - A_{5} t - A_{4} & -A_{11} t^{3} - A_{10} t^{2} - A_{9} t - A_{8} & t^{4}
        \end{array}\right)
\]
of MV cycles. 
% 
More generally, the ideal will be generated by coordinates on the $n\times n$ block in position $(2,3)$ of the MVy slice.
% test
% {{0,1,0,0,0,0,0,0,0,0,0,0},{0,0,1,0,0,0,0,0,0,0,0,0},{0,0,0,1,0,0,0,0,0,0,0,0},{0,0,0,0,A_0,A_1,A_2,A_3,A_4,A_5,A_6,A_7},{0,0,0,0,0,1,0,0,0,0,0,0},{0,0,0,0,0,0,1,0,0,0,0,0},{0,0,0,0,0,0,0,1,0,0,0,0},{0,0,0,0,0,0,0,0,A_8,A_9,A_10,A_11},{0,0,0,0,0,0,0,0,0,1,0,0},{0,0,0,0,0,0,0,0,0,0,1,0},{0,0,0,0,0,0,0,0,0,0,0,1},{0,0,0,0,0,0,0,0,0,0,0,0}}
% \subsection{Case $(a,b) = (n,3)$} 
% Will have 4 MV cycles\dots TODO

\begin{comment}
\subsection{Old, possibly incorrect or mal-calibrated}

\anne{Here are old calculations, in the conventions of CITE PZJ. Explore cxn (of possible deformed algebra structure) to fusion, from the MV cycle POV.}

\begin{example}[$G = \SL_3$ or $\PGL_3$]
    % 
    Let $\lambda = \omega_1 + \omega_2 = (2,1,0)$, $\mu = \omega_3 = (1,1,1)$ and consider the tableaux 
    $$\tau_1 = \young(12,3)\qquad \tau_2 = \young(13,2)$$
    These correspond to the two irreducible components of
    $$\overline{\Gr^{(2,1)}}\cap S^{(1,1)}_-\cong \overline{\OO_{(2,1)}}\cap\n
        = \left\{\begin{bmatrix}
        0 & a & b \\
            & 0 & c \\
            &   & 0 
    \end{bmatrix} : ac = 0 \right\} $$
    as follows
    $$
    X_{\tau_1} = \left\{\begin{bmatrix}
        0 & a & b \\
            & 0 & 0 \\
            &   & 0 
    \end{bmatrix}\right\} \qquad
    X_{\tau_2} = \left\{\begin{bmatrix}
        0 & 0 & b \\
            & 0 & c \\
            &   & 0 
    \end{bmatrix}\right\}
    $$
    The $T\times\CC^\times$-multidegree of $X_{\tau_2}$ is $\hbar + \alpha_1$ (if we normalize like PZJ---else, it'll be $2\hbar + \alpha_1$). The $T\times\CC^\times$-equivariant multiplicity of $X_{\tau_2}$ is 
    $$\frac 1{(\hbar + \alpha_1 + \alpha_2)(\hbar + \alpha_2)}$$
    \end{example}
    % 
    \begin{example}[$G = \SL_2$ or $\PGL_2$]
        For an even simpler example take $\lambda = 2\omega_1 = (2,0)$ and $\mu = \omega_2 = (1,1)$. Then 
        $$\overline{\Gr^{(2,0)}}\cap S^{(1,1)}_-\cong \overline{\OO_{(2,0)}\cap\n}$$
        is irreducible, corresponds to the tableau 
        $$\tau = \young(12)$$
        has multidegree 1 and equivariant multiplicity $\frac 1 {\hbar + \alpha}$.  
\end{example}

% 
\begin{example}
    With Roger we checked using the MVy isomorphism that on the level of MV cycles the fusion rule gives    
    $$
    \young(111,23) \ast \young(12) = \young(11112,23)
    $$
    Does it follow that 
    $$
    \frac 1 {\hbar +\alpha_2} \ast \frac 1 {\hbar +\alpha_1 } = \barD_\hbar (RHS)
    $$
    where (I am not sure of it but) it looks like 
    $$
    \barD_\hbar (RHS) = \frac{7\hbar/2 + \alpha_1 + \alpha_2}{(5\hbar/2 + \alpha_1 + \alpha_2)(2\hbar + \alpha_1)(3\hbar/2 + \alpha_2)}
    $$
\end{example}
\end{comment}

\subsection{Some characters}
No evident patterns. 
% 
{
\begin{table}[ht!]
    \begin{adjustbox}{max width=\textwidth}
    \begin{tabular}{cl}
        $(a,b)$ & $\ch_q$ \\
        \hline
        $(1,1)$ & $q[1|2]+[2|1]$ \\
        & $[1|2]$ \\
        $(1,2)$ & $(q^3+q)[1|2|2]+(q^2+1)[2|1|2]+(q+q^{-1})[2|2|1]$\\
        &$(q+q^{-1})[1|2|2]+[2|1|2]$ \\
        $(1,3)$ & $(q^6+2q^4+2q^2+1)[1|2|2|2]+(q^5+2q^3+2q+q^{-1})[2|1|2|2]+(q^4+2q^2+2+q^{-2})[2|2|1|2]+(q^3+2q+2q^{-1}+q^{-3})[2|2|2|1]$ \\
        & $(q^3+2q+2q^{-1}+q^{-3})[1|2|2|2]+(q^2+2+q^{-2})[2|1|2|2]+(q+q^{-1})[2|2|1|2]$
    \end{tabular}
\end{adjustbox}
\end{table}
}