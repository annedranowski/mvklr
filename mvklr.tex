\documentclass[11pt]{article}
%
\usepackage{bibsty/mvklr}
% 
\title{Deformed DH measure? Deformed MV basis?}
\author{Elie, Anne, Joel}
\date{Last edit:~\today}
%
\begin{document}
% 
\maketitle
% 
\section*{Introduction}
% 
In \cite{baumann2019mirkovic} the authors introduce a measure $\barD:\CC[N]\to\CC(\t)$ defined by 
\begin{equation}
    \label{eq:defbard}
    \barD(f) = \sum_{\vi\in\Seq(\nu)}\langle e_{\vi},f\rangle\barD_{\vi}\qquad f \in \CC[N]_{-\nu}
\end{equation}
where 
\begin{equation}
    \label{eq:bardi}
    \barD_{\vi} = \prod_{k=1}^p\frac{1}{\alpha_{i_1} + \cdots + \alpha_{i_k}} \qquad p = \hgt \nu \,.
\end{equation}
By BKK Proposition 8.4, or Thesis Theorem 3.4.2, it can be realized more invariantly as 
\begin{equation}
    \label{eq:invbard}
    %  \qquad 
     \barD(f)(x) = f(n_x)\,.
\end{equation}
% 
BKK, Proposition A.5 (also, Thesis, Proposition 5.4.4) shows that if $f = b_Z$ is an element of the MV basis indexed by a stable MV cycle of weight $\nu$, i.e.\ $Z\subset \overline{S^\nu_+\cap S^0_-}$, then 
\begin{equation}
    \label{eq:bardaseqm}
    \barD(b_Z) = \varepsilon_{L_0}^T (Z) \,. 
\end{equation}

When $f = c_Y$ is an element of the dual semicanonical basis indexed by an irreducible component $Y$ of $\Lambda(\nu)$
\begin{equation}
    \label{eq:bardaseu} % eu for euler 
    \barD(c_Y) = \sum_{\vi\in\Seq(\nu)}\chi(F_{\vi}(M))\barD_{\vi}
    % \qquad M \in Y\,. 
\end{equation}
where $M$ is a generic point of $Y$. 

We would like an ``asymptotic'' version of $\barD$ which manifests as $T\times\CC^\times$-equivariant multiplicity on elements of the MV basis
$$
\barD_\hbar(b_Z) = \varepsilon_{L_0}^{T\times\CC^\times} (Z) \in \CC(\t,\hbar)
$$
as well as a ``quantum'' version $\barD_q$ which specializes to $\barD_\hbar$ at $q = 1$ or $\hbar = 0$. 
% 

Candidates for deformation are not yet clear. In the files \texttt{a1.tex}, \texttt{a2.tex}, \texttt{a3wt121.tex} and \texttt{Affinesl2.tex} and \texttt{klrw.tex}, we collect examples of characters computed by various means, towards honing in on possible quantum and asymptotic versions of $\barD$.

\section*{Background on MVy}
% 
There are two possibilities for $\barD_\hbar$. To say how we compute them (in type $A$) we need to recall the MVy isomorphism defined in terms of the map 
% $\tilde\Phi : \TT_\mu\cap\cN \to G_1[t^{-1}]t^\mu$ defined by
\begin{equation}
    \begin{split}
        \tilde\Phi : \TT_\mu\cap\cN &\to G_1[t^{-1}]t^\mu \\
        A &\mapsto t^\mu + a(t) \qquad a(t)_{ij} = -\sum A_{ij}^k t^{k-1}
    \end{split}
\end{equation}
where $A_{ij}^k$ denotes the $k$th entry from the left of the $\mu_j\times\mu_i$ block. For the definition of $\TT_\mu$ see?
% We will use it to see what is the $T\times\CC^\times$ multidegree of $Z\subset\Gr_\mu$. 

Let $s\in\CC^\times$ act by loop rotation, and $g = \diag(t_1,\dots,t_m)\in T$ act by conjugation. Another possibility is to allow $g\in T(\cO)$ which would not commute with $\CC^\times$ but intertwine it according to an easy rule. This is the action considered by PZJ. 
We begin with the simpler action of just $T$, so that
$$
(g,s) \cdot \tilde\Phi(A) = {\color{red} s^{\mu}}((s^{-1}t)^\mu + g\cdot a(s^{-1}t)) = t^\mu + g\cdot {\color{red} s^{\mu}} a(s^{-1}t) 
$$
where 
$$
\begin{aligned}
(g\cdot {\color{red} s^{\mu}}a(st))_{ij} &= - t_jt_i^{-1}\sum_{k=1}^{\mu_i} A_{ij}^k s^{k-1{\color{red}+\mu_j}}t^{k-1} \\
&= -t_jt_i^{-1}(A_{ij}^1 s^{\mu_j} + A_{ij}^2 s^{1 + \mu_j}t + \cdots + A_{ij}^{\mu_i}s^{\mu_i + \mu_j - 1}t^{\mu_i - 1}) \\
&= -t_jt_i^{-1}s^{\mu_i + \mu_j - 1}(A_{ij}^1 s^{-\mu_i + 1} + A_{ij}^2 s^{-\mu_i + 2}t + \cdots + A_{ij}^{\mu_i}t^{\mu_i - 1})\,.
\end{aligned}
$$
In the limit $s\to \infty$ the $A_{ij}^{\mu_i}$ term dominates, so the multidegree of $a(t)_{ij}=0$ is 
$$z_j - z_i + (\mu_i + \mu_j - 1)\hbar\,.$$
In particular, the multidegree of zero in $\n$ will be 
$$
\prod_{\beta\in\Delta_+} (\beta + \hbar)
$$
because all $\mu_i = 1$. More generally, the multidegree of zero in $\TT_\mu\cap\n$ will be 
$$
\prod_{1\le i<j\le m} (z_i - z_j + (\mu_i + \mu_j - 1)\hbar)
$$

Now in order to define the asymptotic analogue of $\barD$ on $\CC[N]$ we probably have to deform the ambient ring $\CC[N]$. 

Let's recall how the geometric Satake works. The class of an algebraic cycle $Z\in\irr\overline{\Gr^\lambda\cap S^\mu_-}$ in the Borel--Moore homology of $\overline{\Gr^\lambda\cap S^\mu_-}$ is identified with a vector in $L(\lambda)_\mu$ such that the class of the fixed point $L_\lambda$ is sent to the highest weight vector $v_\lambda$. 

There is a natural map $L(\lambda) \to \CC[N]$ got by sending a vector $v$ to the function $f_v$ such that 
\begin{equation}
    \label{eq:dualorig}
f_v(n) = v_\lambda^\ast(n\cdot v)\,.
\end{equation}
Composing with $\barD$ 
$$
L(\lambda) \to \CC[N] \to \CC(t)
$$ 
we can try to describe $\barD(f_v)$.
% 
Recall that $n_x$ is the unique element of $\CC[N]$ such that $n_x x n_x^{-1} = x + e$ where $e$ is a principal nilpotent. Then 
% 
$$
\barD(f_v) (x) = f_v(n_x) = v_\lambda^\ast (n_x \cdot v)\,. 
$$

By BKK Prop 8.4 (using Lusztig's $\vi$-parametrizations) the test inputs $\{n_x : x \in \lie(T)\}$ are unipotent matrices whose nonzero entries are rational functions in the weights of $T$. We can say that these inputs are simply capturing the info of $\Frac \CC[\lie(T)]$. 
By BKK Corollary 10.6, $\{b_Z(n_x) : x \in \lie(T)\}$ is therefore capturing the weights of $T$ on $Z$ (at zero). 
% 

We need replacements for $n_x$ that are not only made up of rational functions in $\lie(T)$ but also $\lie(\CC^\times)$. 
% 
By the same token, we need replacements for $b_Z$. 
% 

Let's take a couple of steps back. Following \cite{berenstein2006geometric}, let $L(\lambda)$ be the restriction to $\fb$ of the finite-dimensional $\g$-module of highest weight $\lambda$. As a $\fb$-module $L(\lambda)$ can be realized as a quotient of $\cU\n$.
\begin{equation}
    L(\lambda) = \cU \n / I(\lambda) \qquad I(\lambda) = \sum \cU\n e_i^{\lambda(\alpha_i^\vee)+1}
\end{equation} 
The embedding (of $\fb$-modules) defined by Equation~\ref{eq:dualorig} is then just the graded dual of the quotient 
\[
\cU\n \to L(\lambda)    
\]
E.g.\ in \cite{berenstein1996canonical}, it is shown that the dual canonical basis of $\CC[N]$ is thus the specialization at $q = 1$ of the dual canonical basis of the quantized coordinate algebra $\CC_q[N]$. In that paper, the authors consider (the Drinfeld--Jimbo $q$-deformation) $\cU_q(\sl_{r+1})$ and use results of Lusztig and Kashiwara to construct good bases of $\sl_{r+1}$-modules by specializing at $q=1$ good bases of $\cU_q$-modules. Apparently, every $\cU_q$-module has a canonical realization as a subspace of $\CC_q[N]$.  

Does the MV basis come from a basis of a $\cU_q$-module. We know that it can be deformed to a basis of a $\cU_\hbar$-bimodule. What is the relation between these algebras, their representations? 

% But what happens to the $\CC^\times$ action under this map? What happens to the $T$ action for that matter? 

\begin{comment}
{\color{red} Joel says:} what do you mean by the $ \CC^\times$ action and the $ T$ action?  Careful that you are not getting confused with the actioin on the MV cycles (the action used for the multidegrees).  There is a $ T $ action on $L(\lambda)$ and one on $ \CC[N]$, and the map $ L(\lambda) \rightarrow \CC[N]$ is equivariant, up to a character.

It may be helpful to recall that if $Z$ is stable of type $\nu$ then $b_Z$ is unique such that whenever $\nu + \mu \in P_+$ 
$$
t^\mu Z \subset \overline{\Gr^{\nu + \mu }} \Rightarrow b_Z = \Psi_{\nu + \mu}([t^\mu Z])
$$
\end{comment}

% TODO: Uncomment 
% \section{Two rotation actions?}
% 
% In this section we address the two  

% \section{Ben's suggestion} 
% To investigate the special case of  Gelfand--Tsetlin modules and $W$-algebras discussed at the end of \cite{kamnitzer2019category}. I have yet to do any examples. 
% 
% Further references include papers of Losev (\cite{losev2010finite,losev2010quantized,losev2011finite}) and Brundan and Kleschev (\cite{brundan2006shifted,brundan2009blocks}) on $W$-algebras.

\section{Background on KLRW}
% 
TODO 
%
% import bibliography from tex.bib file
%
\bibliographystyle{alpha}
\bibliography{bibsty/mvklr}
%
% for bundling, bbl file contains
%
% \begin{thebibliography}{E-G-S}

% \bibitem[KTWWY19]{kam19}
% Kamnitzer, J., Tingley, P., Webster, B., Weekes, A., \& Yacobi, O. (2019). On category O for affine Grassmannian slices and categorified tensor products. Proceedings of the London Mathematical Society, 119(5), 1179-1233.

% \end{thebibliography}
%
\end{document}