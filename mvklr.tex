\documentclass[11pt]{article}
%
\usepackage{bibsty/mvklr}
% 
\title{Deformed DH measure?}
\author{Elie, Anne, Joel}
\date{Last edit:~\today}
%
\begin{document}
% 
\maketitle
% 
\section*{Introduction}
% 
In \cite{baumann2019mirkovic} the authors introduce a measure $\barD:\CC[N]\to\CC(\t)$ defined by 
\begin{equation}
    \label{eq:defbard}
    \barD(f) = \sum_{\vi\in\Seq(\nu)}\langle e_{\vi},f\rangle\barD_{\vi}\qquad f \in \CC[N]_{-\nu}
\end{equation}
where 
\begin{equation}
    \label{eq:bardi}
    \barD_{\vi} = \prod_{k=1}^p\frac{1}{\alpha_{i_1} + \cdots + \alpha_{i_k}} \qquad p = \hgt \nu \,.
\end{equation}
By BKK Proposition 8.4, or Thesis Theorem 3.4.2, it can be realized more invariantly as 
\begin{equation}
    \label{eq:invbard}
    %  \qquad 
     \barD(f)(x) = f(n_x)\,.
\end{equation}
% 
BKK, Proposition A.5 (also, Anne's Thesis, Proposition 5.4.4) shows that when $f = b_Z$ is an element of the MV basis indexed by a stable MV cycle of weight $\nu$, i.e.\ $Z\subset \overline{S^\nu_+\cap S^0_-}$
\begin{equation}
    \label{eq:bardaseqm}
    \barD(b_Z) = \varepsilon_{L_0}^T (Z) \,. 
\end{equation}

When $f = c_Y$ is an element of the dual semicanonical basis indexed by an irreducible component $Y$ of $\Lambda(\nu)$
\begin{equation}
    \label{eq:bardaseu} % eu for euler 
    \barD(c_Y) = \sum_{\vi\in\Seq(\nu)}\chi(F_{\vi}(M))\barD_{\vi} \qquad M \in Y\,. 
\end{equation}

We would like an ``asymptotic'' $\hbar$-version of $\barD$ which manifests as a $T\times\CC^\times$-equivariant multiplicity on elements of the MV basis:
$$
\barD_\hbar(b_Z) = \varepsilon_{L_0}^{T\times\CC^\times} (Z) \in \CC(\t,\hbar)\,..
$$
% 
The right candidate for deformation is not yet clear. In \texttt{a3casestudy.tex} and \texttt{Affinesl2.tex} we calculate examples towards possible quantum and asymptotic versions of $\barD$.

Recall the MVy isomorphism $\tilde\Phi : \TT_\mu\cap\cN \to G_1[t^{-1}]t^\mu$ defined by
\begin{equation}
    \tilde\Phi(A) = t^\mu + a(t) \qquad a(t)_{ij} = -\sum A_{ij}^k t^{k-1}
\end{equation}
where $A_{ij}^k$ denotes the $k$th entry from the left of the $\mu_j\times\mu_i$ block. We will use it to see what is the $T\times\CC^\times$ multidegree of $Z\subset\Gr_\mu$. Let $s\in\CC^\times$ act by loop rotation, and $g = \diag(t_1,\dots,t_m)\in T$ act by conjugation. Note that if we allowed $g\in T(\cO)$ then these actions would not commute. As it is we set  
$$
(g,s) \cdot \tilde\Phi(A) = {\color{red} s^{\mu}}((s^{-1}t)^\mu + g\cdot a(s^{-1}t)) = t^\mu + g\cdot {\color{red} s^{\mu}} a(s^{-1}t) 
$$
where 
$$
\begin{aligned}
(g\cdot {\color{red} s^{\mu}}a(st))_{ij} &= - t_jt_i^{-1}\sum_{k=1}^{\mu_i} A_{ij}^k s^{k-1{\color{red}+\mu_j}}t^{k-1} \\
&= -t_jt_i^{-1}(A_{ij}^1 s^{\mu_j} + A_{ij}^2 s^{1 + \mu_j}t + \cdots + A_{ij}^{\mu_i}s^{\mu_i + \mu_j - 1}t^{\mu_i - 1}) \\
&= -t_jt_i^{-1}s^{\mu_i + \mu_j - 1}(A_{ij}^1 s^{-\mu_i + 1} + A_{ij}^2 s^{-\mu_i + 2}t + \cdots + A_{ij}^{\mu_i}t^{\mu_i - 1})
\end{aligned}
$$
In the limit $s\to \infty$ the $A_{ij}^{\mu_i}$ term dominates, so the multidegree of $a(t)_{ij}=0$ is 
$$z_j - z_i + (\mu_i + \mu_j - 1)\hbar$$
In particular, the multidegree of zero in $\n$ will be 
$$
\prod_{\beta\in\Delta_+} (\beta + \hbar)
$$
because all $\mu_i = 1$. More generally, the multidegree of zero in $\TT_\mu\cap\n$ will be 
$$
\prod_{1\le i<j\le m} (z_i - z_j + (\mu_i + \mu_j - 1)\hbar)
$$

Now in order to define the asymptotic analogue of $\barD$ on $\CC[N]$ we probably have to deform $\CC[N]$. Why? Let's recall how the geometric Satake works. The class of $Z\in\irr\overline{\Gr^\lambda\cap S^\mu_-}$ in the Borel--Moore homology of $\overline{\Gr^\lambda\cap S^\mu_-}$ is identified with a vector in $L(\lambda)_\mu$ such that the class of the fixed point $L_\lambda$ is sent to the highest weight vector $v_\lambda$. Then $v\in L(\lambda)$ is sent to $f\in\CC[N]$ such that 
$$
f(n) = v_\lambda^\ast(n\cdot v)
$$
But what happens to the $\CC^\times$ action under this map? What happens to the $T$ action for that matter? 


{\color{red} Joel says:} what do you mean by the $ \CC^\times$ action and the $ T$ action?  Careful that you are not getting confused with the actioin on the MV cycles (the action used for the multidegrees).  There is a $ T $ action on $L(\lambda)$ and one on $ \CC[N]$, and the map $ L(\lambda) \rightarrow \CC[N]$ is equivariant, up to a character.

It may be helpful to recall that if $Z$ is stable of type $\nu$ then $b_Z$ is unique such that whenever $\nu + \mu \in P_+$ 
$$
t^\mu Z \subset \overline{\Gr^{\nu + \mu }} \Rightarrow b_Z = \Psi_{\nu + \mu}([t^\mu Z])
$$

\section{Two $\CC^\times$ actions}

\section{Ben's suggestion} 
To investigate the special case of  Gelfand--Tsetlin modules and $W$-algebras discussed at the end of \cite{kamnitzer2019category}. I have yet to do any examples. 

Further references include papers of Losev (\cite{losev2010finite,losev2010quantized,losev2011finite}) and Brundan and Kleschev (\cite{brundan2006shifted,brundan2009blocks}) on $W$-algebras.
%
% import bibliography from tex.bib file
%
\bibliographystyle{alpha}
\bibliography{bibsty/mvklr}
%
% for bundling, bbl file contains
%
% \begin{thebibliography}{E-G-S}

% \bibitem[KTWWY19]{kam19}
% Kamnitzer, J., Tingley, P., Webster, B., Weekes, A., \& Yacobi, O. (2019). On category O for affine Grassmannian slices and categorified tensor products. Proceedings of the London Mathematical Society, 119(5), 1179-1233.

% \end{thebibliography}
%
\end{document}