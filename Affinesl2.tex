\documentclass[11pt]{article}
\usepackage[usenames]{color} %pour la couleur
\usepackage[applemac]{inputenc}
\usepackage{amssymb} %maths
\usepackage{amsmath} %maths
\usepackage{mathabx}
\usepackage{amsthm}
\usepackage[all]{xy}
\usepackage{enumerate}
\usepackage{enumitem}
\usepackage[T1]{fontenc}
\usepackage[left=3.5cm,right=3.5cm,top=3cm,bottom=3.5cm]{geometry}
\usepackage{authblk}
\usepackage{cleveref}
\usepackage{filecontents}
\usepackage{graphics} 
\usepackage{graphicx}
\usepackage{pstricks,pst-node} 
\usepackage{tikz} 
\usepackage{xcolor}
\usepackage{shuffle}
\usepackage{lipsum}


\newcommand\blfootnote[1]{%
  \begingroup
  \renewcommand\thefootnote{}\footnote{#1}%
  \addtocounter{footnote}{-1}%
  \endgroup
}

 \begin{document}
\theoremstyle{plain}
\newtheorem{deftn}{Definition}[section]
\newtheorem{lem}[deftn]{Lemma}
\newtheorem{prop}[deftn]{Proposition}
\newtheorem{thm}[deftn]{Theorem}
\newtheorem{cor}[deftn]{Corollary}
\newtheorem{conj}[deftn]{Conjecture}
\newtheorem{Question}[deftn]{Question}
\newtheorem{Fact}{Evidence}
\newtheorem{assump}{Assumption}[section]
\renewcommand{\theassump}{\Alph{assump}}
\newtheorem{conjintro}{Conjecture}
\newtheorem{thmintro}{Theorem}

\theoremstyle{definition}
\newtheorem{ex}[deftn]{Example}
\newtheorem{rk}[deftn]{Remark}


\newcommand{\CQ}{\mathcal{C}_Q}
\newcommand{\CZ}{\mathcal{C}_{\mathbb{Z}}}
\newcommand{\Atn}{\mathcal{A}_t (\mathfrak{n})}
\newcommand{\CN}{\mathbb{C}[N]}


\title{KLR modules in type $A_l^{(1)}$}
\author{Elie Casbi and Anne Dranowski and Joel Kamnitzer}
 
\date{Last edit: \today}

\maketitle

\section{Type $A_1^{(1)}$}

Consider the unipotent cell of $\CN$ associated with the reduced word $(1,0,1,0)$. Then the $\bar{D}$ of the simple modules of the  initial seed (i.e. the standard seed corresponding to the same reduced word) are given by 
$$ \frac{1}{\alpha_1} \quad  \frac{1}{\alpha_1^2(\delta + \alpha_1)} \quad  \frac{1}{\alpha_1^3(\delta + \alpha_1)^2(2 \delta + \alpha_1)} \quad \frac{1}{\alpha_1^4(\delta + \alpha_1)^3(2 \delta + \alpha_1)^2(3 \delta + \alpha_1)} . $$ 
Conjecturally, when considering the reduced word $(1,0,1,0, \ldots , 1,0)$ with $n$ occurrences of $0$ and $1$ we will get fractions of the form 
$$  \frac{1}{\alpha_1^n(\delta + \alpha_1)^{n-1} \cdots ((n-1) \delta + \alpha_1)} . $$ 

\section{Type $A_2^{(1)}$}

The Weyl group of type $A_2^{(1)}$ is generated by $s_0,s_1,s_2$ satisfying braid relations. 
Consider the unipotent cell of $\CN$ associated with the reduced word $(1,2,0,1,2,0)$. Let $M_1, \ldots , M_6$ denote the corresponding KLR modules. Assuming things go as in finite simply-laced type, i.e. $\bar{D}(M_i) = 1/P_i$ with the $P_i$ satisfying the relations 
$$ P_j P_{j_{-}} = \beta_j \prod_{k < j < k_{+}} P_k $$
we obtain the following:
$$
\begin{aligned}
  & P_1 = \alpha_1 , \quad  P_2 = \alpha_1(\alpha_1 + \alpha_2) , \quad   P_3 = (\delta + \alpha_1)(\alpha_1 + \alpha_2)\alpha_1^2 , \\
  & P_4 = (\delta + \alpha_1 + \alpha_2)(\delta + \alpha_1)(\alpha_1 + \alpha_2)^{2}\alpha_1^2 , \\
  & P_5 = (2 \delta  + \alpha_1)(\delta + \alpha_1 + \alpha_2)(\delta + \alpha_1)^{2}(\alpha_1 + \alpha_2)^{2}\alpha_1^{3} , \\
  & P_6 = (2 \delta  + \alpha_1)(2 \delta + \alpha_1 + \alpha_2)(\delta + \alpha_1 + \alpha_2)^{2}(\delta + \alpha_1)^{2}(\alpha_1 + \alpha_2)^{3}\alpha_1^{3}
\end{aligned}
$$
Probably for the reduced word $(s_1s_2s_0)^n$ we shall obtain two families of polynomials given by 
$$ Q_n = ((n-1) \delta  + \alpha_1)((n-1) \delta + \alpha_1 + \alpha_2) \cdots (\delta + \alpha_1 + \alpha_2)^{n-1}(\delta + \alpha_1)^{n-1}(\alpha_1 + \alpha_2)^{n}\alpha_1^{n} $$
and 
$$ R_n = ((n-1) \delta  + \alpha_1)((n-2) \delta + \alpha_1 + \alpha_2) \cdots (\delta + \alpha_1)^{n-1}(\alpha_1 + \alpha_2)^{n-1}\alpha_1^{n} . $$
On the example $n=2$, we have $P_1=R_1, P_2=Q_1, P_3=R_2, P_4=Q_2, P_5=R_3, P_6=Q_3$.
First note that 
$$ R_{n+1} / Q_n = \alpha_1(\alpha_1 + \delta) \cdots (\alpha_1 + n \delta) . $$
We also observe the following:
$$   
\begin{aligned}
  & P_3 / P_{1} = \alpha_1(\alpha_1 + \alpha_2)(\delta + \alpha_1), \quad  P_3 / P_2 = \alpha_1(\delta + \alpha_1), \\
  & P_5 / P_4 = \alpha_1(\delta + \alpha_1)(2 \delta + \alpha_1), \quad  P_5 / P_3 = \alpha_1(\alpha_1 + \alpha_2)(\delta + \alpha_1)(\delta + \alpha_1 + \alpha_2)(2 \delta + \alpha_1)
\end{aligned}
$$    
\bigskip    
     
So $P_3 / P_1$ recovers the first eqm of weight $(2,1)$ you computed in type $A_2$. It seems that one can actually recover the whole first family for this weight, by performing $P_2 / P_1 \times R_n / Q_{n-1}$. I do not know how to obtain the second family. 

Another question: do the polynomials 
$$Q_{n+1} / Q_n = \alpha_1 \cdots (\alpha_1 + n \delta)(\alpha_1 + \alpha_2) \cdots (\alpha_1 + \alpha_2 + n \delta)$$
{\color{blue}Anne: Yes I think this one coincides with Lusztig datum $(n,0,n)$. Have to think about the one below.}

and 
$$R_{n+1} / R_n =   \alpha_1 \cdots  (\alpha_1 + n \delta)(\alpha_1 + \alpha_2) \cdots ((n-1) \delta + \alpha_1 + \alpha_2) $$
coincide with the denominators of certain type $A_2$ eqms? 

 \bigskip
 
  For any non-negative integers $(k,l)$, define the following  polynomial:
  \begin{align*}
   Q_{k,l} & = \alpha_1^k (\alpha_1 + \delta)^{k-1} \cdots (\alpha_1 + (k-1)\delta) \alpha_2^l (\alpha_2 + \delta)^{l-1} \cdots (\alpha_2 + (l-1)\delta) \\ & \qquad \qquad \qquad  (\alpha_1 + \alpha_2)^{k+l} (\alpha_1 + \alpha_2 + \delta)^{k+l-1} \cdots (\alpha_1 + \alpha_2 + (k+l-1)\delta) .  
     \end{align*}
  It is also convenient to slightly extend this definition by setting
  $$ Q_{k,-1} := \alpha_1^k (\alpha_1 + \delta)^{k-1} \cdots (\alpha_1 + (k-1)\delta) (\alpha_1 + \alpha_2)^{k-1} (\alpha_1 + \alpha_2 + \delta)^{k-2} \cdots (\alpha_1 + \alpha_2 + (k-2)\delta) . $$
Now fix a couple of integers $(n,m)$ with $n \in 2 \mathbb{N}_{\geq 1}$ and $m$ arbitrary, and consider the reduced element $(s_1s_2s_0)^n(s_2s_1s_0)^m$ in the Weyl group of type $A_2^{(1)}$. Then the polynomials obtained for the corresponding standard seed are the $Q_{k,l} , (k,l) \in \{(k, \epsilon - 1) , 1 \leq k \leq n+1 \} \cup \{ (n + \epsilon , l) , 0 \leq l \leq m \}$. 
The unipotent cell $\mathbb{C}[N((s_1s_2s_0)^n(s_2s_1s_0)^m)]$ contains many other standard seeds. It seems that the form of the polynomials is preserved under mutations from one the others. Now note that these polynomials are in fact entirely determined by the order of $\alpha_1, \alpha_2, \alpha_1 + \alpha_2$. In other words we consider the following map:
$$
 \begin{array}{ccc}
      KP(\mathfrak{sl}_3) & \longrightarrow & \mathbb{Z}[\alpha_1, \alpha_2] \\
      (a,b,c) & \longmapsto & Q[a,b,c]
 \end{array}
$$
where
\begin{align*}
 Q[a,b,c] & :=  \alpha_2^a (\alpha_2 + \delta)^{a-1} \cdots (\alpha_2 + (a-1)\delta) (\alpha_1 + \alpha_2)^{b} (\alpha_1 + \alpha_2 + \delta)^{b-1} \cdots \\ &   \qquad  \qquad \qquad  \qquad \qquad  (\alpha_1 + \alpha_2 + (b-1)\delta) \alpha_1^c (\alpha_1 + \delta)^{c-1} \cdots (\alpha_1 + (c-1)\delta). 
 \end{align*}
Then conjecturally the $\bar{D}$ of the flag minors of the unipotent cell $\mathbb{C}[N((s_1s_2s_0)^n(s_2s_1s_0)^m)]$ should be of the form $ 1 / Q[k,k+l,l]$. Now one has to consider certain specific quotients of these polynomials. Below are the few examples I have obtained so far. 
\begin{enumerate}
    \item $Q[0,n,n]/Q[0,n-1,n-1] = \alpha_1 (\alpha_1 + \delta) \cdots (\alpha_1 + (n-1) \delta) (\alpha_1 + \alpha_2)(\alpha_1 + \alpha_2 + \delta) \cdots (\alpha_1 + \alpha_2 + (n-1) \delta)$.
    \item $Q[0,n-1,n]/Q[0,n-1,n-1] = \alpha_1 (\alpha_1 + \delta) \cdots (\alpha_1 + (n-1) \delta)$.
    \item $Q[n,n,0]/Q[n-1,n-1,0] = \alpha_2 (\alpha_2 + \delta) \cdots (\alpha_2 + (n-1) \delta) (\alpha_1 + \alpha_2)(\alpha_1 + \alpha_2 + \delta) \cdots (\alpha_1 + \alpha_2 + (n-1) \delta)$.
    \item $Q[n,n-1,0]/Q[n-1,n-1,0] = \alpha_2 (\alpha_2 + \delta) \cdots (\alpha_2 + (n-1) \delta)$.
    \item $\left( Q[0,1,1]/Q[0,n-1,n] \right) \times \left( Q[0,n-1,n]/Q[0,n-1,n-1] \right) = \alpha_1(\alpha_1 + \alpha_2)  (\alpha_1 + \delta) \cdots (\alpha_1 + (n-1) \delta)$. 
\end{enumerate}

The first four correspond to the $T \times \mathbb{C}^{*}$-equivariant multiplicities of the MV cycles with respective Lusztig data $(0,n,0), (0,0,n), (n,0,n)$ and $(n,0,0)$. These precisely correspond the powers of each of the four cluster variables in $\mathbb{C}[N]$ of type $A_2$. The fifth correspond to the $T \times \mathbb{C}^{*}$-equivariant multiplicities of the first familty of Lusztig data for the weight $n \alpha_1 + \alpha_2$. 

\end{document}