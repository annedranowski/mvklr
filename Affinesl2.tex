\documentclass[11pt]{article}
\usepackage[usenames]{color} %pour la couleur
\usepackage[applemac]{inputenc}
\usepackage{amssymb} %maths
\usepackage{amsmath} %maths
\usepackage{mathabx}
\usepackage{amsthm}
\usepackage[all]{xy}
\usepackage{enumerate}
\usepackage{enumitem}
\usepackage[T1]{fontenc}
\usepackage[left=3.5cm,right=3.5cm,top=3cm,bottom=3.5cm]{geometry}
\usepackage{authblk}
\usepackage{cleveref}
\usepackage{filecontents}
\usepackage{graphics} 
\usepackage{graphicx}
\usepackage{pstricks,pst-node} 
\usepackage{tikz} 
\usepackage{xcolor}
\usepackage{shuffle}
\usepackage{lipsum}


\newcommand\blfootnote[1]{%
  \begingroup
  \renewcommand\thefootnote{}\footnote{#1}%
  \addtocounter{footnote}{-1}%
  \endgroup
}

 \begin{document}
\theoremstyle{plain}
\newtheorem{deftn}{Definition}[section]
\newtheorem{lem}[deftn]{Lemma}
\newtheorem{prop}[deftn]{Proposition}
\newtheorem{thm}[deftn]{Theorem}
\newtheorem{cor}[deftn]{Corollary}
\newtheorem{conj}[deftn]{Conjecture}
\newtheorem{Question}[deftn]{Question}
\newtheorem{Fact}{Evidence}
\newtheorem{assump}{Assumption}[section]
\renewcommand{\theassump}{\Alph{assump}}
\newtheorem{conjintro}{Conjecture}
\newtheorem{thmintro}{Theorem}

\theoremstyle{definition}
\newtheorem{ex}[deftn]{Example}
\newtheorem{rk}[deftn]{Remark}


\newcommand{\CQ}{\mathcal{C}_Q}
\newcommand{\CZ}{\mathcal{C}_{\mathbb{Z}}}
\newcommand{\Atn}{\mathcal{A}_t (\mathfrak{n})}
\newcommand{\CN}{\mathbb{C}[N]}


\title{KLR modules in type $A_1^{(1)}$}
  \author{Elie Casbi and Anne Dranowski and Joel Kamnitzer}
 
     \date{}

\maketitle

\section{A couple of computations}

Consider the unipotent cell of $\CN$ associated with the reduced word $(1,0,1,0)$. Then the $\bar{D}$ of the simple modules of the  initial seed (i.e. the standard seed corresponding to the same reduced word) are given by 
$$ \frac{1}{\alpha_1} \quad  \frac{1}{\alpha_1^2(\delta + \alpha_1)} \quad  \frac{1}{\alpha_1^3(\delta + \alpha_1)^2(2 \delta + \alpha_1)} \quad \frac{1}{\alpha_1^4(\delta + \alpha_1)^3(2 \delta + \alpha_1)^2(3 \delta + \alpha_1)} . $$ 
Conjecturally, when considering the reduced word $(1,0,1,0, \ldots , 1,0)$ with $n$ occurrences of $0$ and $1$ we will get fractions of the form 
$$  \frac{1}{\alpha_1^n(\delta + \alpha_1)^{n-1} \cdots ((n-1) \delta + \alpha_1)} . $$ 

 \end{document}